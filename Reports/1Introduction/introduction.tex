\chapter{INTRODUCTION}
% (20% of Proposal/report Length)
\pagenumbering{arabic}

\section{Background}
\vspace{10pt}
In the present context of time, natural language processing (NLP) and text analytics involve using machine learning algorithms and narrow artificial intelligence (AI). Natural Language Processing is an area in computer science that studies the interactions and acts as an interface between computers and human languages. The analysis of languages can be done manually, but the technology continues to strive in aspects of Machine and Deep learning communities. In this new world of cognitive computing, humans and machines work better together than individually. Collaborative learning emphasizes the benefits of using digital tools to help individuals learn together and improve their quality of life through advanced technology.\\
\\
Machine learning for NLP and text analytics involves a set of statistical techniques for identifying parts of speech, entities, sentiments, and various aspects within the text. The techniques can be expressed as a model that is then applied to other texts also called supervised machine learning. In supervised machine learning, a batch of text documents is tagged or annotated with examples of what the machine should look for and how it should interpret that aspect. This means feeding a machine learning model an annotated dataset in a short time.\\
\\
Most importantly, Machine learning means teaching the machine what it needs to know or how it needs to act. The main task is to create a learning framework and provide properly formatted, relevant, and clean data for the machine to learn from and adapt to a pattern. A model is a mathematical representation of the input key. A machine learning model has been seen before the model can use this prior learning to evaluate all new cases.\\
\\
Machine Learning models are very good at performing single tasks such as determining the sentiment polarity of a document or a part of speech for a given word. However, models cannot work alone on tasks that require layers of interpretation. Suppose for an NLP system, first, we want to know whether there are any entities and then sentiments associated with those as well as identify industry; then this system will require multiple models to interact with each other to create a successful outcome.

\section{Motivation}
\vspace{10pt}
The main motivation for this project is to create an application that can in many ways act as a catalyst to assist in the advancement of Natural Language Processing in Nepali (Devanagari) text. This project will aim to facilitate more advanced work in the field of data science and artificial intelligence by expanding the scope of Nepali language processors. With a target audience of about 13.7 million people worldwide who communicate in the Nepali language, we can increase the impact of Nepali News outlets and resources. Although our ground only covers 0.17 percent of the entire world’s population, it cannot be overlooked that this step towards the advancement of technology and AI in Nepali texts can expand the scope of the Nepali language, expand its outreach to the world and be more beneficial for the native speakers.\\
\\
The main motivation behind this project is to build a model that is capable of accurately identifying named entities, primarily categorized as: person, organization and location. The long term effect of this project is to create a proficient NER application that can be further expanded for other prospects such as facilitating potential applications. Additionally, this project should help to establish a reliable source of Nepali Devanagari script based language processing .

\section{Problem Statement}
\vspace{10pt}
 The problem statements:
 \begin{itemize}
  \item Named Entity Recognition (NER) has been thoroughly researched and facilitated in most of the prominent resource-rich languages.
  \item There are several reliable methods and data sets available as 
    there has been a lot of development in NER research.
  \item 	However, NER is mostly unexplored in languages with limited 
    linguistic resources and a narrow study focus, like Nepali which are becoming scarce as western languages are taking over the world.
    \item There is a requirement to create a Nepali language-based NLP 
    application and expand the scope of Nepali language.
\end{itemize}
 
 
\section{Project Objectives}
\vspace{10pt}
\begin{itemize}
    \item To develop an accurate Nepali Named Entity Recognition (NER) system for names, locations, and organizations.
\end{itemize}
   
\section{Scope of Project}
\vspace{10pt}

Natural Language Processing (NLP)’s named entity recognition (NER) sub-component seeks to identify the textual existence of entities that fall under a predetermined set of categories, such as “PERSON”, “LOCATION”, “ORGANIZATION”. NER considers the surrounding context of an entity to determine its appropriate type. For example, “SHIVAM(in nepali)” could refer to the person’s name or the organization, and NER can differentiate based on the context. NER can be applied to process real-time data streams, making it useful for applications like social media monitoring, news analysis, and chatbots. NER can be a crucial step in information extraction tasks. \\
\\
Despite its remarkable capabilities, Named Entity Recognition (NER) has inherent limitations that shape its scope. NER's accuracy heavily relies on context; ambiguous terms can lead to misclassifications. Variations in spelling, context, or language can challenge NER's precision. Domain-specific entities or newly coined terms might not be recognized accurately. Additionally, NER's performance diminishes for languages with complex morphology or sparse training data. The need for substantial training data and fine-tuning can hinder its applicability in low-resource languages. Moreover, NER's inability to comprehend intricate relationships between entities can limit its effectiveness in complex contexts. While NER is a powerful tool, these constraints underscore the importance of considering its limitations within its broader scope.


 
\newpage
\section{Potential Project Applications}
\vspace{10pt} % Adjust the value as needed

\begin{itemize}
\item  \textbf{Location Tracking}

\textbf{Contextual Understanding} : Identifying locations in news articles provides contextual understanding, helping readers and analysts grasp the geographical relevance of the reported events.


\textbf{Impact Assessment} : Knowing where events occur allows for a more accurate assessment of their impact, especially in cases of disasters, conflicts, or significant developments that may be location dependent.

\textbf{Spatial Trends} : Analyzing the distribution of news events across different locations can reveal spatial trends, enabling the identification of patterns, correlations, or concentrations of specific types of news in certain areas.

\textbf{Geopolitical Insights} : Location tracking aids in gaining geopolitical insights by highlighting regions of frequent news coverage, which can be indicative of geopolitical hotspots, emerging trends, or areas of global significance.

\item \textbf{General public and public figure mentions}
\vspace{10pt} % Adjust the value as needed

\textbf{Information Retrieval} : NER enhances search engine capabilities by accurately identifying and indexing person names. There is precision in the search results especially regarding mentions of specific people. It provides tailored results as per the interest of the user.

\textbf{Market \& Academic Research} : NER assists in analyzing people or customers,   their reviews, forums and other mentions providing an insight into consumer sentiment and preference. At the same time identifying and categorizing different people provides research for analysis, academic studies, and aids competitive intelligence.

\textbf{Human Resources and Recruitment} : Incorporating a person’s name helps recognize people most likely to match certain descriptions or meet desired criteria. NER helps identify people fit for a job through resume parsing and assists in background checks all through an individual's name.\\

\item \textbf{Organization referencing}
\vspace{10pt} % Adjust the value as needed

\textbf{Investment \& Financial Analysis} : NER helps extract and analyze business articles, reports, and market research regarding a particular organization. This helps understand the market trend and gives industry insights. It also aids in investment analysis and reflects the impact of an organization on the market.

\textbf{Performance Evaluation} : Information extraction of news coverage provides details on the organization’s activities, achievements, and challenges. The impact of global trends on the organization reflects its credibility and performance.
\end{itemize}






\vspace{10pt}

% \section{Requirement Analysis}
% \vspace{10pt}
% \subsection{Economic Feasibility} 
% Currently, this project is entirely software-based, making it economically feasible. The 
% costs associated with this project include a high gpu computer, launching on the internet (if launched/hosted). Apart from this, there is no need for 
% any complex hardware, making this project a cost-effective solution.

% \subsection{Technical Analysis}
% Our computing platform was Google Colab, which free resources made simple connectivity with Google Drive for data access.
% The pre-trained, BERT-based multilingual architecture forms the basis of the model creation process. We used a self-labeled test set that was specially modified to our requirements and an open-source dataset for training to fine-tune this model. The whole training process took place inside Google Colab, removing the requirement for complex local installations. We were able to easily train the model using the available tools and upload our dataset straight from Google Drive. Google Colab's user-friendly setting and group collaboration help us.

% In conclusion, the project successfully demonstrated the technical feasibility of the SABDA-NER model. Google Colab's accessibility and freely available resources greatly facilitated the development and training process, highlighting its potential for future NLP projects.

% \subsection{Schedule Feasibility}
% This project can be completed within the allotted time frame for the Major Project and 
% will be carried out as per the timestamps marked in the Gantt chart.

% \subsection{Legal Feasibility}
% All the software tools and algorithms that are being used in this project is ‘Free to use’ 
% or ‘Free to modify’. Since the majority of the implementation is conducted 
% independently, there are no legal obstacles that would render the project legally 
% infeasible.


\section{Originality of Project}
% Originality of Project (Relatable to the Research Gap)
% There are few research and projects on  Named Entity Recognition in the Nepali language and less data set. We have prepared and trained Nepali dataset to obtain the categories, such as “PERSON”, “LOCATION”, “ORGANIZATION” as named entities. Use of the latest language model BERT developed by Google gives the predictable result which help to suppress the results of previous models like SVM model, Hidden Markov Model with the n-gram technique for extracting POS tags, BiLSTM, BiLSTMCNN, BiLSTM-CRF, and BiLSTM-CNN-CRF with different word embedding. This project provides the user interface where user can input nepali text to get the named entity as mentioned above.

 \vspace{10pt}
 
In the field of understanding and identifying specific elements in Nepali text, our project is a game-changer. We've taken the initiative to create a unique dataset and train it meticulously. This dataset allows us to extract important information like names of people ("PERSON"), places ("LOCATION"), and organizations ("ORGANIZATION") from Nepali content.

What makes our project stand out is the use of a powerful tool called BERT, developed by Google. This advanced language model not only improves the accuracy of our results but also outperforms older models like SVM and Hidden Markov Model with n-gram techniques for POS tag extraction, BiLSTM, BiLSTMCNN, BiLSTM-CRF, and BiLSTM-CNN-CRF with various word embeddings.

But we're not just about complex algorithms. Our project is user-friendly with an interface that allows users to input Nepali text effortlessly. The outcome? A precise extraction of named entities, focusing specifically on categories such as "PERSON," "LOCATION," and "ORGANIZATION." By integrating the latest language modeling advancements, our project not only surpasses previous models but reshapes the landscape of Nepali Named Entity Recognition. Welcome to a new era where the beauty of language meets cutting-edge technology, providing an unmatched user experience.

\newpage
\section{Organisation of Project Report}

The material in this project report is organised into seven chapters. After this introductory chapter introduces the problem topic this research tries to address, chapter 2 contains the literature review of vital and relevant publications, pointing toward a notable research gap. Chapter 3 describes the methodology for the implementation of this project. Chapter 4 provides an overview of what has been accomplished. Chapter 5 contains some crucial discussions on the used model and methods. Chapter 6 mentions pathways for future research direction for the same problem or in the same domain. Chapter 7 concludes the project shortly, mentioning the accomplishment and comparing it with the main objectives.