% \section{Literature Review of Base Paper- III}
% \onehalfspacing

% \begin{table}[H]
% \begin{adjustbox}{max width=\textwidth}
% \begin{tabular}{p{8.69cm}p{0.07cm}p{7.98cm}p{0.07cm}}
% \hline
% \multicolumn{4}{|p{16.82cm}|}{{ \textbf{Author(s)/Source: }
% Li, Michael Hong Gang
% }} \\ 
% \hline
% \multicolumn{4}{|p{16.82cm}|}{{ \textbf{Title: }
% Objective Vision-based Assessment of Parkinsonism and Levodopa-induced Dyskinesia in Persons with Parkinson’s Disease
% } 
% } \\ 
% \hline
% \multicolumn{4}{|p{16.82cm}|}{{ \textbf{Website: \url{https://tspace.library.utoronto.ca/handle/1807/77844}
% }}} \\ 
% \hline
% \multicolumn{2}{|p{8.76cm}}{{ \textbf{Publication Date:
% June, 2017
% }}} & 
% \multicolumn{2}{|p{8.05cm}|}{{ \textbf{Access Date: 
% December, 2021
% }}} \\ 
% \hline
% \multicolumn{2}{|p{8.76cm}}{{ \textbf{Publisher or Journal: }}{ 
% University of Toronto
% }} & 
% \multicolumn{2}{|p{8.05cm}|}{{ \textbf{Place: }} { 
% School of Graduate Studies
% }} \\ 
% \hline
% \multicolumn{2}{|p{8.76cm}}{{ \textbf{Volume: }
% n/a
% }} & \multicolumn{2}{|p{8.05cm}|}{{ \textbf{Issue Number: }
% n/a
% }} \\ \hline
% \multicolumn{4}{|p{16.82cm}|}{{ \textbf{Author's position/theoretical position: }} { 
% Master’s Student
% }} \\ \hline
% \multicolumn{4}{|p{16.82cm}|}{{ \textbf{Keywords:}} { 
% Computer vision; Deep learning; Disease management; Health monitoring; Parkinson's disease
% }} \\ \hline
% \multicolumn{4}{|p{16.82cm}|}{{ \textbf{\underline {Important points, notes, quotations } 
% \hfill \underline {Page No.}} }
% \begin{enumerate}
% 	 \vspace{-0.2cm} \item { 
% 	 Development of human pose estimation benchmark from the PD evaluation dataset.
% 	 \hfill \textbf{90}}
% 	\vspace{-0.5cm}\item { 
% 	Two DL  methods have been tested for Pose Estimation
% 	\hfill \textbf{22}}
% 	\vspace{-0.5cm}\item { 
% 	Video-based features as clinically actionable information for neuroscientists. 
% 	\hfill \textbf{58}}
% 	\vspace{-0.5cm}\item {
% 	Markerless computer-based visual system to complement existing clinical practice 
% 	\hfill \textbf{94}
% 	}\vspace{-0.5cm}\end{enumerate}
% } \\ 
% \hline
% \multicolumn{4}{|p{16.82cm}|}{{ \textbf{Essential Background Information: }} {
% Computerized assessment can be a solution to the need of frequent automatic assessment of Parkinson's Disease signals without the help of a doctor.
% }} \\ \hline
% \multicolumn{4}{|p{16.82cm}|}{{ \textbf{Overall argument or hypothesis: }
% Computer vision methods are capable of tracking the position and movement of the body in clinical PD assessment videos such that scores calculated can be coorelated with clinical socres.
% }} \\ \hline
% \multicolumn{4}{|p{16.82cm}|}{{ \textbf{Conclusion: }} { 
% The results show that the pose estimation algorithm can extract relevant information about the motor signals of Parkinson's disease from video assessments and the calculated scores correlates well.
% }} \\ 
% \hline
% \multicolumn{1}{|p{9.69cm}}{{ \textbf{Supporting Reasons}}} & 
% \multicolumn{2}{p{8.05cm}|}{} \\ 
% \hhline{~~~}
% \multicolumn{1}{|p{8.69cm}}{{ \textbf{1. }
% Models using movement features extracted from videos as input correlated to clinical ratings.
% }} & 
% \multicolumn{2}{p{8.05cm}|}{{ \textbf{2. }
% Could detect the presence of PD/LID and also predict its severities.
% }} \\ 
% \hhline{~~~}
% \multicolumn{1}{|p{8.69cm}}{{ \textbf{3. }
% Objective movement features could bring a new scoring paradigm in PD assessment.
% }} & 
% \multicolumn{2}{p{8.05cm}|}{{ \textbf{4. }
% Evaluated latest human pose estimations algorithms in clinical assessment videos.
% }} \\ 
% \hhline{~~~}
% \multicolumn{1}{|p{8.69cm}}{{ \textbf{5. }
% Exploration of the motion features that could be extracted from video analysis.
% }} & 
% \multicolumn{2}{p{8.05cm}|}{{ \textbf{6. }
% Identification of the important features of the movement for good model performance.
% }} \\ 
% \hhline{~~~}
% \multicolumn{1}{|p{8.69cm}}{{ \textbf{7. }
% The regression model predicted severity with a high correlation with the clinical score.
% }} & 
% \multicolumn{2}{p{8.05cm}|}{{ \textbf{8. }
% Although consumer-grade video cameras were used, results are promising.
% }} \\ 
% \hline
% \multicolumn{3}{|p{16.74cm}|}{{ \textbf{Strengths of the line of reasoning and supporting evidence: }
% Objective evaluation has been done with strong  evidence. E.g., evaluation of correlation coefficients has confirmed the results with a sufficient degree. Also, their results with benchmarking datasets have strong mathematical ground.
% }} \\ 
% \hline
% \multicolumn{3}{|p{16.74cm}|}{{ \textbf{Flaws in the argument and gaps or other weaknesses in the argument and supporting evidence: }
%  Because of pose estimation from a single 2D image, information loss occurs when the patient is moving perpendicularly to the camera plane and largely influence the results.
% } 
% } \\ 
% \hline
% \end{tabular}
% \end{adjustbox}
% \end{table}