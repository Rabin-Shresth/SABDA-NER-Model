\chapter{CONCLUSION}


% Conclusion: (1 Page)

% a. Summarize the key supporting ideas discussed throughout the project

% b. Relate back to the project objectives and discuss about their fulfillment

% c. Offer final impression on the project's central idea


% This initiative aimed to solve the following research question:

% \textit{. . . .}


% This project has contributed to the area of . . .  in the following ways to achieve the goal:

% \begin{itemize}
%     \item ...

%     \item ...
    
%     \item ...
    
% \end{itemize}

% With the above contributions,  this project has shown high hopes . . .


In conclusion, the objective of the project have been successfully achieved. The implementation of BERT based NER model to detect the nepali named entity sets people (”PERSON”), places (”LOCATION”), and organizations (”ORGANIZATION”) from Nepali content.\\

Use of a powerful language model called BERT this advanced language model not only improves the accuracy of our
results but also outperforms older models like SVM and Hidden Markov Model with
n-gram techniques for POS tag extraction, BiLSTM, BiLSTMCNN, BiLSTM-CRF, and
BiLSTM-CNN-CRF with various word embeddings.
We have integrated the model with web user interface where user can provide nepali text and get result.\\
 
 Throughout the project, several challenges were encountered, such as data collection, 
data annotation, training, and computational issues. However, these challenges were addressed successfully. Additionally, future enhancements were 
proposed, such as better dataset collection, standardizing annotations and additional named entity like date, months medical terms etc. This project offers an innovative solution for natural language processing field Named Entity recognition for Nepali language that can be scaled to make great use of it.
