%==============================Abstract Page=================================================
\chapter*{ABSTRACT}
\addcontentsline{toc}{chapter}{ABSTRACT}
\thispagestyle{plain} 
% Abstract (200 to 250 Words)
\vspace{10pt}
 
% Natural Language Processing (NLP) is a computerized approach to analyzing text data, combining theories and technologies. It's a leading field in research and technology, advancing computer science and AI. NLP aims to replicate human-like language processing, covering syntactic and semantic understanding. Using mathematical models, NLP addresses language complexities, creating advanced systems for language-related tasks. It's an interdisciplinary field, integrating linguistics, logic, computer science, and AI. 
% Named Entity Recognition (NER), a pivotal NLP task, involves the identification and classification of Named Entities (NEs) within unstructured text, thereby assigning them to predefined categories like Person Names, Locations, and Organizations. This process bestows structure upon raw textual data, extracting valuable insights and enabling downstream analyses. While NER has garnered significant attention in languages rich with linguistic resources, such as English, its comprehensive exploration remains limited for languages like Nepali—characterized as resource-poor. The dearth of labeled data and linguistic tools for Nepali presents a challenge for applying conventional techniques. Nonetheless, the advent of Deep Learning has ushered in a promising era for NER in such languages, potentially bridging the gap between languages with abundant resources and those with constrained linguistic assets. As NER continues to evolve, it stands as a testament to NLP's overarching mission to decode and replicate the nuances of human communication through technological prowess. The strides taken in this realm not only enhance our understanding of languages across the globe but also underscore NLP's significance in reshaping how computers comprehend and interact with the written word.
The proposed project aims to create a Nepali language-based Named Entity Recognition (NER) application for identifying and analyzing Nepali texts. Successfully developed a Natural Language Processing (NLP) application that recognizes and classifies Nepali named entities, focusing on Personal names, Organisation names, and Locations. NLP, a computerized approach to text analysis, combines theories and technologies. NER, a crucial NLP task, involves identifying and classifying Named Entities (NEs) in unstructured text, assigning them to predefined categories like Person Names, Locations, and Organizations. Despite challenges in resource-poor languages like Nepali, marked by a lack of labeled data and linguistic tools, Deep Learning offers a promising solution, potentially bridging the gap between languages with abundant resources and those with constrained linguistic assets. As NER evolves, it underscores NLP's mission to decode and replicate human communication nuances through technological advancement, shaping how computers comprehend and interact with written language.



\par
\textit{Keywords: Deep Learning, NER,  NEs, NLP}
 
%=============================================================================================
% (a) Inclusion of three to four Keywords (Lexicographical Order)
\newpage