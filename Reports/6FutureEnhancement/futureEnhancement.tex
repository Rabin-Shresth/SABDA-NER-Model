\chapter{FUTURE ENHANCEMENTS}

% Enhancements: (1 to 2 Pages)

% a. Mention approaches that were not attempted, but could have been experimented with for better results

% b. Recommended a research path for future researchers that may embark on a similar research topic

Following are the region of improvements that could be worked upon in the future:
\begin{itemize}
    \item \textbf{Dataset Collection}
The quality of the dataset has a direct impact on the accuracy of the trained models. 
Therefore, it is crucial to collect nepali data sets in a very accurate environment best data collection methods can be implemented.
  It is also recommended to explore elevations other than 
ground level to diversify the dataset. To further improve the accuracy of the model, 
collecting data from various sources such as news corpus  and goverment data in nepali, diverse dataset, should be emphasized.
\item \textbf{Dataset Annotation}
The dataset annotation process could be standardized by defining which part of the setence should be in correct named entity region. This will ensure that the labelling process is consistent and avoid 
irregularities that could lead to problems in the model's ability to detect regions of 
named entity in nepali text.


\item \textbf{Domain-Specific Training}

Fine-tune the NER model on domain-specific data to enhance its performance on particular industries or subject matters. we can ensure that the model is tailored to the unique language and entities present in specific domains.
 
 
 \item \textbf{Multilingual Support}

Expand language support by training and evaluating the model on multilingual datasets. This will helps the NER system recognize named entities in languages beyond the initially trained language, facilitating broader applicability.

 
\item \textbf{Contextual Understanding}

Enhance the model's contextual understanding by incorporating more advanced contextual embeddings or leveraging state-of-the-art transformer-based architectures, such as BERT, GPT, or RoBERTa, which have demonstrated success in capturing intricate contextual relationships.\\

\item \textbf{Entity Linking}

Integrate entity linking capabilities to associate recognized entities with relevant knowledge bases or databases. This enriches the information extracted from the text and provides additional context for downstream applications. Develop mechanisms to adapt to changes in named entities over time. As language evolves and new entities emerge, the NER system should be capable of learning and recognizing these changes dynamically.
 
\item \textbf{Evaluation Metrics}

Develop and employ more nuanced evaluation metrics that consider the importance or rarity of specific named entities, providing a more comprehensive assessment of the model's performance.
\end{itemize}

 